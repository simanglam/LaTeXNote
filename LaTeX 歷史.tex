\chapter{一些有趣的故事}

在文章正式開始之前,我想要先撈叨一些關於 \LaTeX\ 的歷史,因為 \LaTeX\ 的發展史十分有趣,也是極致的駭客精神與開源精神結合的產物,如果對這一段沒有興趣,可以直接跳過這段,直接從教學開始讀起,這並不會影響你對 \LaTeX\ 的學習過程,如果你想了解這段歷史,就讓我們跳上時光機,回到 20 世紀吧!

\section{\LaTeX 歷史}

時間回到 20 世紀,在高德納教授 (Donald Ervin Knuth) 在撰寫他的著作 《The Art of Computer Programming》時,因為當時電腦排版還很粗糙,高德納教授覺得書商把自己的著作排得太難看了,所以他決定自己撰寫一個電腦排版軟體,來拯救自己的著作,於是 \TeX\ 就被發明了。

\TeX\ 是一個低階的排版系統,它可以利用簡單的命令來執行排版任務,也可以將一連串的低階命令封裝成高階命令,以節省時間。通常沒有人會直接使用底層的 \TeX\ 來排版,他們會載入預先定義好的格式,格式包含了大量的\TeX\ 命令,涵蓋了所有的排版細節,就像 CSS 之於 HTML 一樣,高德納教授就自己寫了一個格式 ``Plain \TeX '' 並使用它來進行排版工作。

Plain \TeX\ 被撰寫之後大大地減低了使用 \TeX\ 的難度,但對於普通人來說,Plain \TeX\ 還是過於艱澀難懂,好在 ``Leslie B. Lamport'' 教授因為他的出版需求,所有撰寫出了 \LaTeX\ 這個格式,並且無私的將其授權給所有人使用。至此 \LaTeX\ 就開始了他稱霸學術排版的征途。

\section{我該不該用\LaTeX ?}

相信剛接觸到 \LaTeX\ 的人都會苦惱於這個問題,這種問題通常是來源於對 \LaTeX\ 的認識不深,所以才會產生此種疑問,這一篇就是要協助各位釐清思緒的,先來說說 \LaTeX\ 的優點:

\begin{itemize}
\item 格式穩定,在不同的電腦間不易跑版
\item 數學方程式
\item 非所見即所得,使作者可以專注於文件內容而非排版細節。
\end{itemize}
\pskip
通常情況下只要原始碼一樣 \LaTeX\ 產出的結果也會一樣,這對高度要求版面一制的工作尤其有利,且因為實際上進行排版的是電腦而不是人類,人類只是在描述物件的屬性,所以在撰寫書籍時可以將內容與排版細節分開,以章節標題為例,在 Word 裡標題需要經過放大字體、置中、粗體來創建,但在 \LaTeX\ 中只需要用 \verb|\section{標題}|  \LaTeX\ 就會自動執行放大字體、置中、粗體的步驟,使我們可以更加專注在文章內容。

上文講了一些 \LaTeX\ 的優點,看起來很美好,但沒有事情是完美無缺的,所以下文就要分析 \LaTeX\ 的缺點:

\begin{itemize}
\item 非所見即所得,細節調整要求來回編譯。
\item 學習曲線陡峭
\end{itemize}

非所見即所得既是優點也是缺點,雖然可以讓作者更專注於文章內容,但要微調細節時就會需要不斷地來回編譯。\LaTeX\ 陡峭的學習曲線是學習 \LaTeX\ 時最大的挑戰,但在度過初期的困難之後,你就會發現 \LaTeX\ 之美了。

看完了以上的文章,如果你還對 \LaTeX\ 有興趣,就請接續往下讀吧,但如果你明天就要交作業,可是你今天才接觸到 \LaTeX\ 還是請你暫緩腳步,先將自己的作業交出,再來學習 \LaTeX\ 吧。
