\chapter{自定義命令與環境}

在看完了這麼多的 package 後,你會發現有時候為了追求美觀,可能會需要利用到大量的命令去調整輸出,如果只使用一次那還好說,但如果需要在文件中不斷地使用到這個功能,不斷重複的程式碼一來會增加 Debug 的困難度、二來還會降低效率,為了解決這種問題 \LaTeX 提供了以下四個命令來協助使用者自定義命令與環境。

\begin{itemize}
\item \verb|\newcommand{cmd}[必選參數的數量]{definition}|
\item \verb|\renewcommand{cmd}[必選參數的數量]{definition}|
\item \verb|\newenvironment{env}[必選參數的數量]{before env}{after env}|
\item \verb|\renewenvironment{env}[必選參數的數量]{before env}{after env}|
\end{itemize}

\LaTeX 是是一個大小寫敏感的程式語言,所以 \verb|\LaTeX| 與 \verb|\latex| 是有所差別的,另外 \LaTeX 在命名上不能使用數字、符號等等,所有的命令皆只能由字母組成。另外除了定義新的命令外\LaTeX 提供了兩個以 re 開頭的命令,以讓我們改寫已被定義過的命令或環境。

\section{自定義命令}

在初步了解過如何自定義命令後就可以上手試試了,如果我們在排版中需要重複使用一個複雜的樣式,例如需要一個黃底、紅字、粗體的樣式來特別標注重點,如果正常使用應該會需要將三個命令套在一起使用,先不說方便性,使用的次數一多還會降低可讀性,所以我們可以將這一連串的命令定義成一個新的命令:

\begin{tcblisting}{}
\newcommand{\important}[1]{%定義名為 \important 的命令
\colorbox{yellow}{\textcolor{red}{\textbf{#1}}}
}
這是一個非常\important{重要的公告}
\end{tcblisting}

可以看到我將 \verb|\important| 定義為一串命令,如此一來即可利用\verb|\important| 來代替一長串的命令,而在定義的過程中 \# 1代表的是必選參數一,若這個可選命令有多個必選參數,也必須以此類推,將 \# 1 一路加到 \# 9 為止,需要注意 \LaTeX 的必選參數是不可以超過 9 個的。

\section{自定義環境}

定義環境與定義命令相似,不同的點在於環境需要定義環境開始與結束時的命令,例如想要一個特殊的環境來區分引用的文字,可以從想要的呈現結果開始著手,我想要環境的上下留一點白、環境內的文字要改成斜體、還要一個必選參數將來源標示出來,請看以下的範例:

\begin{tcblisting}{listing only}
\newenvironment{poetry}[1]{%
\newcommand{\poetryfer}{#1}
\vskip6pt plus 2pt minus 2pt
\let\oldparindent\parindent
\setlength{\parindent}{0pt}
\setlength{\leftskip}{12pt}\itshape}{
\\ -\poetryfer 
\vskip6pt plus 2pt minus 2pt
\setlength{\parindent}{\oldparindent}
}
\end{tcblisting}

\begin{poetry}{邱妙津《蒙馬特遺書》}
世界總是沒有錯的,\\
錯的是心靈的脆弱性,\\
我們不能免除於世界的傷害,\\
於是我們就要長期生著靈魂的病。
\end{poetry}

上面是這個範例的結果,在這個範例中,我們定義了 poetry 環境,在環境開始之前會先由 vskip 留白,接者再利用 \verb|\let| 命令將原先的縮排長度暫存起來,而後將縮排長度改成零,最後將整段的縮進設為 12pt 字體設為斜體。在環境結束的時候,先換到下一行再將來源標示出來,要注意在第二格括號內是不能使用必選參數的,但可以先在第一個括號中儲存成命令,在第二個括號中再利用命令的方式使用。

這就是在 \LaTeX 中自定義環境與命令的方式,善用這個技巧可以確保排版的一致性也可以減少所需使用的命令數。