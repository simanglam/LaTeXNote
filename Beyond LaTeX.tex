\chapter{Beyond LaTeX}

終於完稿了,當初會想寫下這本書可說全部都是意外,事情得要從一個地科作業開始說起,當時在寫地科作業的我正被「如何在 page 內加入數學方程式」而困擾著,於是我打開了 page 內建的插入方程式功能,只見一行大字出現在視窗內「請使用 mathml 或 \LaTeX\ 來插入數學方程式」,這就是我遇見 \LaTeX\ 的過程。

後來我就開始學習 \LaTeX\ ,在學習的過程中我發現跟 \LaTeX\ 有關的中文資料只有兩種,不是簡體字就是有一定年份的資料,除了這之外就全部都是英文資料了,中文資料的缺乏讓我開始思考我可以做什麼來改善這個情況,於是我開始撰寫了這本書,希望可以為臺灣的 \LaTeX\ 社群增添一份可用的資料。

在撰寫途中不禁感嘆自己對 \LaTeX\ 瞭解的不足,同時也驚嘆於 \LaTeX\ 強大的排版功能,也讓我撰寫本書的決心更加強烈,雖然我只是一個高中生,這也只是一個自主學習的計畫,但我也是有力量的,只要我將我的力量貢獻出,某些需要幫助的人就一定能收到,最後希望各位都可以在撰寫 \LaTeX\ 一帆風順。

\begin{flushright}
周造麟-Si manglam\\
Email: qwer09214@gmail.com\\
2022/10/10 寫下
\end{flushright}