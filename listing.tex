\chapter{listing}

要想在 \LaTeX 中列出程式碼顯然不能手動在每一個命令前都手動加一個\verb|\textbackslash|,一來這樣過於耗時且造成可讀性降低,二來這樣也沒有完善的語法標示,要解決這個困難可以利用一些 package 來解決。

\section{\LaTeX 內建}

如果不利用其他 package ,\LaTeX 本身就有內建這個功能,我們可以利用\verb`\verb| |` 來將命令給列出來:

\begin{tcblisting}{listing side text}
用 \verb|\begin{center}| 來將文字置中
\end{tcblisting}

這通常用於在文章段落中穿插的時候,如果有一長串的程式碼要展示就不適合用這個方法了,這種時候可以改用 verbatim 環境,這兩個就像是在數學章節中提到的展式數式與隨文數式的差別一樣。

\begin{tcblisting}{}
\begin{verbatim}
\newcounter{example}
\newenvironment{example}{
\refstepcounter{example}
\textbf{Example.\theexample}\ 
}{\\}
\end{verbatim}
\end{tcblisting}

可以看到這樣單獨的產生了一個區塊來展示程式碼,可是這樣是沒有語法突顯,看起來會稍微有些痛苦,想要改善這種情況就只能額外用 package 了。

\section{listings}

首先要介紹的是 listing,

\begin{tcblisting}{listing side text}
\begin{lstlisting}
\begin{itemize}
\item 1
\item 2
\item 3
\end{itemize}
\end{lstlisting}
\end{tcblisting}

你會看到上面的例子也沒有改善多少,這是因為我們還沒定義程式碼的樣式。

\begin{tcblisting}{listing side text}
\begin{lstlisting}[language={[LaTeX]TeX}, commentstyle=\color{red} ,keywordstyle=\color{blue}, numbers=left]
\begin{itemize}
\item 1
\item 2
\item 3
\end{itemize}
%一個不知道為什麼的列表
\end{lstlisting}
\end{tcblisting}

這樣看起來就好許多了,可是每一次都打這一大長串也不方便,所以可以利用 \verb`\lstset `來設定默認的參數。

\begin{tcblisting}{listing only}
\lstset{
    language={[LaTeX]TeX},
    basicstyle=\sffamily,
    numbers=left,
    numberstyle=\scriptsize,
    frame=tb,
    tabsize=4,
    commentstyle=\color{blue},
    keywordstyle=\color{red},
    morekeywords={ce,draw,node,foreach,in,chemfig,bond,href,hologo,
    ifthenelse,addplot,addplot3,coordinates}
}
\end{tcblisting}

language 是設定程式語言的類型,basicstyle 是設定列出來成果的格式,numberstyle 是控制數字的格式,commentstyle 是控制註解的格式,keywordstyle 是控制關鍵字的格式,morekeywords 則是可以自行加入星的關鍵字。

\section{minted}

但上面的方法有一個問題,就是他只能標記出有被設定過的關鍵字,有時候多少會有點不方便,於是有人將 Pygment 與 \LaTeX\ 結合起來做成了 minted 這個 package,在使用 minted 之前請先確保自己的電腦內有 Pygment詳細的下載方式請參考下面這篇文章。<https://clay-atlas.com/blog/2020/02/10/python-chinese-tutorial-package-pygments-code-highlight/>

minted 提供了 \verb`\begin{minted}[參數]{語言}`......\verb`\end{minted}` 來輸出程式碼

\begin{tcblisting}{listing side text}
\begin{minted}{latex}
\begin{itemize}
\item 1
\item 2
\item 3
\end{itemize}
\end{minted}
%一個不知道為什麼的列表
\end{tcblisting}

可以看到這樣好看很多,如果想要微調輸出格式可以在利用以下的參數。

\begin{tabular}{cc}
參數 & 含義 \\\hline\hline
lineos & 顯示程式碼行數 \\\hline
bgcolor & 背景顏色 \\\hline
numbers & 顯示程式碼行數(可指定位置) \\\hline
mathescape & 可以在 minted 環境中直接輸入數學方程式 \\\hline
escapeinside & 設定跳拖字符 \\\hline
breaklines & 可不可將程式碼換行 \\\hline
\end{tabular}

\begin{tcblisting}{listing side text}
\begin{minted}[linenos, breaklines, mathescape, escapeinside=| |]{latex}
\begin{itemize}
\item 1
\item 2
\item 3
\end{itemize}
$\Sigma_{k=100} \sin(k^\circ)$
|\textcolor{red}{ABC}|
\end{minted}
\end{tcblisting}

如果想要使用別種配色,Pygment 有內建許多不同的 style 可供選擇,只要使用 \verb`\usemintedstyle{style}` 選擇即可。

\begin{tcblisting}{listing side text}
%\usemintedstyle{vim}
\begin{minted}[linenos, breaklines, mathescape, escapeinside=| |]{latex}
\begin{itemize}
\item 1
\item 2
\item 3
\end{itemize}
$\Sigma_{k=100} \sin(k^\circ)$
|\textcolor{red}{ABC}|
\end{minted}
\end{tcblisting}

這樣就可以更改樣式了,詳細的樣式與支持的語言,請參考 Pygment 的官網。