\chapter{數學}

\TeX\ 原先被發明的原因便是因為不滿當時數學方程式的排版,而這麼多年過去了依舊沒有任何軟體可以在數學排這方面與 \TeX 平起平坐,以 \TeX 為核心的 \LaTeX 當然也繼承了這份優勢,至今依舊是論文排版的首選工具,其中被用到最多的功能就是數學排版,這也是在使用 \LaTeX 的過程中該去鑽研的一個部分。

\section{基礎使用}

想要在 \LaTeX 中打出數學方程式必須要先進入數學模式,想要進入數學模式可以使用以下幾種方式:

\begin{itemize}
\item \verb|$......$| 
\item \verb|\(.....\)|
\item math 環境
\item \verb|\[.....\]|
\item displaymath 環境
\item equation 環境
\end{itemize}

以上幾種方式都會進入數學模式,會有這麼多種方式主要還是因為有多種需求,主要會是隨文(math inline mode)與展式(math display mode)這兩種不同的樣式,隨文會將方程式放在段落中的文字之間,而展示則會賃開空間將數學方程式獨立展示,這可以用於強調一個重要的公視會數學推導過程。

\begin{figure}[htp]
\begin{minipage}{0.5\textwidth}
隨文數式:
\( F = G\frac{m_1m_2}{r^2}\)
\end{minipage}
\begin{minipage}{0.5\textwidth}
展式數式:
\[ F = G\frac{m_1m_2}{r^2}\]
\end{minipage}
\caption{展式數式與隨文數式}
\end{figure}


可以看到這兩種不同的樣式的區別,但這兩者的原始碼皆相同,只是由於樣式的不同所以結果不同。

\section{數學符號與實例}

在一般的輸入法中,是沒有辦法直接打出數學符號的,就算可以也不建議這樣使用,因為很多時候都需要依照前後的符號來調整間距,所以還是建議利用命令的方式來輸入數學符號。

\section{分數}

分數是利用 \verb|\frac{}{}| 來進行排版的:

\begin{tcblisting}{listing side text}
\[
V = \frac{\Delta S}{T}
\]
\end{tcblisting}

