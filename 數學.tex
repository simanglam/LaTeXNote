%\chapter{數學}
%
%\TeX\ 原先被發明的原因便是因為不滿當時數學方程式的排版,而這麼多年過去了依舊沒有任何軟體可以在數學排這方面與 \TeX 平起平坐,以 \TeX 為核心的 \LaTeX 當然也繼承了這份優勢,至今依舊是論文排版的首選工具,其中被用到最多的功能就是數學排版,這也是在使用 \LaTeX 的過程中該去鑽研的一個部分。
%
%\section{基礎使用}
%
%想要在 \LaTeX 中打出數學方程式必須要先進入數學模式,想要進入數學模式可以使用以下幾種方式:
%
%\begin{itemize}
%\item \verb|$......$| 
%\item \verb|\(.....\)|
%\item math 環境
%\item \verb|\[.....\]|
%\item displaymath 環境
%\item equation 環境
%\end{itemize}
%
%以上幾種方式都會進入數學模式,會有這麼多種方式主要還是因為有多種需求,主要會是隨文(math inline mode)與展式(math display mode)這兩種不同的樣式,隨文會將方程式放在段落中的文字之間,而展示則會賃開空間將數學方程式獨立展示,這可以用於強調一個重要的公視會數學推導過程。
%
%\begin{figure}[htp]
%\begin{minipage}{0.5\textwidth}
%隨文數式:
%\( F = G\frac{m_1m_2}{r^2}\)
%\end{minipage}
%\begin{minipage}{0.5\textwidth}
%展式數式:
%\[ F = G\frac{m_1m_2}{r^2}\]
%\end{minipage}
%\caption{展式數式與隨文數式}
%\end{figure}
%
%
%可以看到這兩種不同的樣式的區別,但這兩者的原始碼皆相同,只是由於樣式的不同所以結果不同。
%
%\section{數學符號與實例}
%
%在一般的輸入法中,是沒有辦法直接打出數學符號的,就算可以也不建議這樣使用,因為很多時候都需要依照前後的符號來調整間距,所以還是建議利用命令的方式來輸入數學符號。
%
%\section{分數}
%
%分數是利用 \verb|\frac{}{}| 來進行排版的:
%
%\begin{tcblisting}{listing side text}
%\[
%V = \frac{\Delta S}{T}
%\]
%\end{tcblisting}
%
\chapter{數學}

\LaTeX\ 很大的一部分功用是排版科學相關文章,而佔最大宗的還是數學相關的文章,因為 \LaTeX\ 有著平易近人的數學輸入法以及足夠大的談鋞,至今扔是學術界慣用的排版軟體。

\section{基本概念}

最簡單的用法是將方程式用 \verb`$......$` 包起來,這樣可以在行內插入數學方程式

\begin{tcblisting}{listing side text}
畢氏定理$C =\sqrt{A^2 + B^2} $
\end{tcblisting}

但當方程式很複雜、或非常重要,讓你需要為他特別清出空間,好彰顯這個方程式的重要性,這時可以使用 \verb`\[......\]` 把方程式包起來

\begin{tcblisting}{listing side text}
畢氏定理:
\[C =\sqrt{A^2 + B^2}\]
相當的重要
\end{tcblisting}

雖然這兩者在輸入上沒有任何的差別,但在輸出上還是會有些許的不同

\begin{tcblisting}{listing side text}
這是隨文數式:$\Sigma^{60}_{k=31}\sin^2k^\circ$\\
這是展示數式:
\[ \Sigma^{60}_{k=31}\sin^2k^\circ \]
\end{tcblisting}

可以看到上下標的位置有所改變

\section{基礎使用}

先從最簡單的四則運算開始說起,除了乘、除的符號需要用 \verb`\times` 與 \verb`\div` 表示以外,其他的運算子都不需要使用命令來表示。

\begin{tcblisting}{listing side text}
$A + B - C \times D \div E = F$
\end{tcblisting}

如果想要輸出分數,需要使用 \verb`\frac{分子}{分母}` 輸出

\begin{tcblisting}{listing side text}
$\frac{a}{b}\\
(\frac{a}{b})^2$
\end{tcblisting}

上面的例子有一個問題,第二行的括號會看起來太小,這時候可以利用 \verb`\left(......\right)` 來讓 \LaTeX\ 自動調整括號的大小。

\begin{tcblisting}{listing side text}
$\left(\frac{a}{b}\right)^2$
\end{tcblisting}

今天的內容有涉及到美國數學家協提供的 \textbf{amssym, amsfonts} 與 \textbf{amsmath} ,若有涉及到這些 package 的應用,我會在下面特別標注,如果沒有標註就是\LaTeX\ 基本的使用。

\section{各種的應用}

基本的函數都是用反斜槓加函數名稱的方式輸出

\begin{tabular}{cccc}
\hline
\verb`\sin` & \verb`\cos` & \verb`\tan` & \verb`\cot` \\\hline
\verb`\arccos`  & \verb`\arcsin` & \verb`\arctan` & \verb`\sec` \\\hline
\verb`\csc`  & \verb`\exp` & \verb`\log` & \verb`\deg` \\\hline
\verb`\lim` & \verb`\inf`  & \verb`\min` & \verb`\max` \\\hline
\end{tabular}

想要使用其他字體嗎?\LaTeX\ 提供了以下幾種字體

\begin{tabular}{cc}
\hline
字體|結果 \\\hline\hline
\verb`\mathrm{ABCabc123}` & $\mathrm{ABCabc123}$ \\\hline
\verb`\mathit{ABCabc123}` & $\mathit{ABCabc123}$ \\\hline
\verb`\mathnormal{ABCabc123}` & $\mathnormal{ABCabc123}$ \\\hline
\verb`\mathcal{ABCabc123}` & $\mathcal{ABCabc123}$ \\\hline
\end{tabular}

數學模式的輸出皆為斜體,可以用 \verb`\mathrm{}` 轉為正體,如果想在數學模式中加粗字體,可以利用 amsmath 提供的 \verb`\boldsymbol` 命令

\begin{tcblisting}{listing side text}
$
\mu ,\boldsymbol{\mu}\\
\delta ,\boldsymbol{\delta}
$
\end{tcblisting}

空心粗體則需要 \textbf{amsfonts} 提供的 \verb`\mathbb{}` 命令

\begin{tcblisting}{listing side text}
$
x > 1 and x \in \mathbb{R}
$
\end{tcblisting}

如果想要將某個公式的推導過程寫下,可以利用 \textbf{amsmath} 提供的 align 環境

\begin{tcblisting}{listing side text}
\begin{align}
x &= 20 +5\\
x &= 25
\end{align}
\end{tcblisting}

在想要對齊的地方用 \& 指定即可,實際上的使用方式就與表格類似,如果不想要編號,使用帶星號的 \verb`align*` 即可

\begin{tcblisting}{listing side text}
\begin{align*}
x &= 20 +5\\
x &= 25
\end{align*}
\end{tcblisting}

如果需要輸出矩陣,可以使用 \verb`matrix` 環境

\begin{tcblisting}{listing side text}
$
\begin{matrix}
3 & 0\\
0 & 3
\end{matrix}
$
\end{tcblisting}

但這樣就只是一些對齊的的數字,所以我們可以利用以下的方式來輸出含有小括號的矩陣

\begin{tcblisting}{listing side text}
\[
\left(\begin{matrix}
2 & 0\\
0 & 2
\end{matrix}\right)
\]
\end{tcblisting}

或者使用由 amsmath 提供的 pmatrix 環境

\begin{tcblisting}{listing side text}
\[
\begin{pmatrix}
2 & 0\\
0 & 2
\end{pmatrix}
\]
\end{tcblisting}

不只是小括號,也可以使用方括號、花括號

\begin{tcblisting}{listing side text}
\[
\begin{bmatrix}
2 & 0\\
0 & 2
\end{bmatrix}
\begin{Bmatrix}
2 & 0\\
0 & 2
\end{Bmatrix}
\]
\end{tcblisting}

甚至是行列式也可以利用這個方法輸出

\begin{tcblisting}{listing side text}
\[
\begin{vmatrix}
2 & 0\\
0 & 2
\end{vmatrix}
\begin{Vmatrix}
2 & 0\\
0 & 2
\end{Vmatrix}
\]
\end{tcblisting}

如果想要輸出聯立方程式,可以利用 amsmath 提供的 cases 環境

\begin{tcblisting}{listing side text}
\[
\begin{cases}
x &= 1\\
y &= 3x + 9
\end{cases}
\]
\end{tcblisting}

當然,我所列出的例子只是滄海一粟,實際上還有更多的可能性,但由於我很少利用這部分的功能,所以我只能簡單地把我知道的使用方式全都寫出,更進一步的使用方式可以參考這些文章。