\chapter{化學}

\LaTeX\ 也能拿來排版化學相關的事物,但我們需要借用 mhchem 與 chemfig 的力量

\begin{tcblisting}{listing only}
\usepackage{mhchem}
\usepackage{chemfig}
\end{tcblisting}

\section{化學式 \& 化學反應式}

化學式與化學反應式利用 mhchem 提供的 \verb`\ce{}` 就可以達成了

\begin{tcblisting}{listing side text}
\ce{H2O}\\
\ce{H2O2}\\
\ce{NO-}
\end{tcblisting}

如果需要質量數可以用以下的方式

\begin{tcblisting}{listing side text}
\ce{^235_98U}\\
\ce{^2_1H}\\
\ce{^4_2He}
\end{tcblisting}

\verb`^` 代表上標 \verb`_` 代表下標,也可以打出分子內離子的氧化態

\begin{tcblisting}{listing side text}
\ce{Fe^{II}Fe^{III}2O4}
\end{tcblisting}

計量化學也可以利用相同的方式

\begin{tcblisting}{listing side text}
\ce{2H2O}\\
\ce{1/2H2O}\\
\ce{(1/2)H2O}\\
\ce{$n$H2O}
\end{tcblisting}

化學反應式只需要加入 \verb`+` 或 \verb`->` 等等就好了

\begin{tcblisting}{listing side text}
\ce{H2O2 -> H2O + O2}
\end{tcblisting}

如果涉及到沈澱或產生氣體可以利用單獨的 \verb`^` 跟單獨的小寫 v,可逆反應則更改箭頭的樣式即可

\begin{tcblisting}{listing side text}
\ce{^ v}\\
\ce{A <=> B}\\
\ce{CaCO_3 + HCl <=> CaCl_2 v + H_2O + CO_2 ^}
\end{tcblisting}

如果需要加催化劑,可以用箭頭後加中括號的方式達成

\begin{tcblisting}{listing side text}
\ce{A ->[text above][text below] B]}\\
\ce{H2O2 ->[MnO2] H2O + O2 ^}\\
$\ce{x Na(NH4)HPO4 ->[\Delta] (NaPO3)_x + x NH3 ^ + x H2O}$
\end{tcblisting}

下圖是 mhchem 可以使用的箭頭種類

\section{結構式}

結構式需要借助 chemfig 提供的 \verb`\chemfig{}` 命令

\begin{tcblisting}{listing side text}
\chemfig{H-O-H}
\end{tcblisting}

你可能會想要調整角度,在 \verb`-` 後加[]可以解決這個問題,chemfig 可以接受預設角度、絕對角度與相對角度的輸入,預設角度就直接在括號內加入數字,預設是 0 ,之後每增加 1 角度增加 45 度,絕對角度需要在數字前加入\verb`:`,相對角度則是加入\verb`::` 

\begin{tcblisting}{listing side text}
\chemfig{A-[1]-[2]-[3]-[4]-[5]-[6]-[7]-[8]}
\chemfig{A-[:45]-[:90]-[:135]-[:180]-[:225]-[:270]-[:315]-[:360]}
\chemfig{A-[::+45]-[::+45]-[::+45]-[::+45]-[::+45]-[::+45]-[::+45]-[::+45]}
\end{tcblisting}

如果想要畫多邊形可以利用下面的技巧

\begin{tcblisting}{listing side text}
\chemfig{C*5(-A-B-C-D-E-F)}\\
\chemfig{[:18]C*5(-A-B-C-D-E-F)}
\end{tcblisting}

如果你真的想做點什麼複雜的東西,我建議你可以參考奈米小人。
