\chapter{tcolorbox}

這篇要介紹 tcolorbox 這個 package,他是利用 \TikZ 來繪製好看的 colorbox,同時兼具了強大的客製化功能,讓書本內需要的部分可以強調。

\section{基礎使用}

在 tcolorbox 中提供了 tcolorbox 的環境來創造 colorbox,這個環境有許多的可選參數,調整這些可選參數可以調整整個 colorbox 的樣式,

\begin{tcblisting}{listing side text}
\begin{tcolorbox}[colback=gray!30!white, colframe=gray!30!white]
這是一個 colorbox。
\end{tcolorbox}
\end{tcblisting}

\begin{description}
\item[colback]
\item[colbacklower]
\item[colframe]
\item[coltext]
\item[subtitle style]
\item[fonttitle]
\item[boxrule]
\end{description}

\begin{description}
\item[colupper]
\item[collower]
\item[fontupper]
\item[fontlower]
\end{description}

%參數 & 含義 \\\hline\hline
%colback & 底色 \\\hline
%colbacklower & 下半部分的底色 \\\hline
%colframe & 邊匡顏色 \\\hline
%coltitle & title 欄的底色 \\\hline
%colupper & 上半部分文字的顏色 \\\hline
%collower & 下半部分文字的顏色 \\\hline
%coltext & 文字顏色 \\\hline 
%subtitle style & title 欄的樣式 \\\hline
%boxrule & 邊匡粗細 \\\hline
%fonttitle & 標題文字的樣式 \\\hline
%fontupper & 上半部分文字的樣式 \\\hline
%fontlower & 下半部分文字的顏色 \\\hline

需要注意的是 colbacklower 需要搭配其他命令才可使用,之後會介紹到,如果想要設定一個預設值可以利用`\tcbset{}`來完成。

\subsection{標題與副標題}

可以用`[title=title]`為他加入標題

\begin{tcblisting}{listing side text}
\begin{tcolorbox}[title=Title]
This is a colored box with a title.
\end{tcolorbox}
\end{tcblisting}

也可以用`\tcbsubtitle{Subtitle}`來插入副標題

\begin{tcblisting}{listing side text}
\begin{tcolorbox}[title=Title]
This is a colored box with a title.
\tcbsubtitle{Subtitle}
And subtitle.
\end{tcolorbox}
\end{tcblisting}

\subsection{上下分段}

如果你想要將一個 box 分成兩段可以利用 \verb`\tcblower`

\begin{tcblisting}{listing side text}
\begin{tcolorbox}
Upper Box
\tcblower
Lower Box
\end{tcolorbox}
\end{tcblisting}

這樣預設會是上下兩段,可以利用 sidebyside 改成左右各佔一半

\begin{tcblisting}{listing side text}
\begin{tcolorbox}[sidebyside]
Upper Box
\tcblower
Lower Box
\end{tcolorbox}
\end{tcblisting}

\subsection{更多}

但 tcolorbox 可不只有這樣,你可以利用 \verb`\tcbuselibrary{}` 去調用一些延伸功能,例如調用 skin 可以讓上下兩段的顏色分開設定

\begin{tcblisting}{listing side text}
%\tcbuselibrary{skins}
\begin{tcolorbox}[skin=bicolor, sidebyside, colback=gray!30!white,colbacklower=gray!5!white]
Bicolor
\tcblower
Bicolor
\end{tcolorbox}
\end{tcblisting}

當然除了上述的技巧之外,tcolorbox 還有許多用處我沒有講到,如果想要好好的研究可以參考他的使用手冊<連結>。