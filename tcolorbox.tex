\chapter{tcolorbox}

今天要介紹的是 tcolorbox,它提供了一個簡單的產生高度客製化 color box 的方式。

\section{基礎使用}

tcolorbox 提供了 tcolorbox 這個環境供我們建立 colorbox

\begin{tcblisting}{listing side text}
\begin{tcolorbox}
This is a colored box.
\end{tcolorbox}
\end{tcblisting}

\subsection{Style}

\begin{tabular}{cc}
參數 & 含義 \\\hline\hline
colback & 底色 \\\hline
colbacklower & 下半部分的底色 \\\hline
colframe & 邊匡顏色 \\\hline
coltitle & title 欄的底色 \\\hline
colupper & 上半部分文字的顏色 \\\hline
collower & 下半部分文字的顏色 \\\hline
coltext & 文字顏色 \\\hline 
subtitle style & title 欄的樣式 \\\hline
boxrule & 邊匡粗細 \\\hline
fonttitle & 標題文字的樣式 \\\hline
fontupper & 上半部分文字的樣式 \\\hline
fontlower & 下半部分文字的顏色 \\\hline
\end{tabular}

需要注意的是 colbacklower 需要搭配其他命令才可使用,之後會介紹到,如果想要設定一個預設值可以利用`\tcbset{}`來完成。

\subsection{標題與副標題}

可以用`[title=title]`為他加入標題

\begin{tcblisting}{listing side text}
\begin{tcolorbox}[title=Title]
This is a colored box with a title.
\end{tcolorbox}
\end{tcblisting}

也可以用`\tcbsubtitle{Subtitle}`來插入副標題

\begin{tcblisting}{listing side text}
\begin{tcolorbox}[title=Title]
This is a colored box with a title.
\tcbsubtitle{Subtitle}
And subtitle.
\end{tcolorbox}
\end{tcblisting}

\subsection{上下分段}

如果你想要將一個 box 分成兩段可以利用 \verb`\tcblower`

\begin{tcblisting}{listing side text}
\begin{tcolorbox}
Upper Box
\tcblower
Lower Box
\end{tcolorbox}
\end{tcblisting}

這樣預設會是上下兩段,可以利用 sidebyside 改成左右各佔一半

\begin{tcblisting}{listing side text}
\begin{tcolorbox}[sidebyside]
Upper Box
\tcblower
Lower Box
\end{tcolorbox}
\end{tcblisting}

\subsection{更多}

但 tcolorbox 可不只有這樣,你可以利用 \verb`\tcbuselibrary{}` 去調用一些延伸功能,例如調用 skin 可以讓上下兩段的顏色分開設定

\begin{tcblisting}{listing side text}
%\tcbuselibrary{skins}
\begin{tcolorbox}[skin=bicolor, sidebyside, colback=gray!30!white,colbacklower=gray!5!white]
Bicolor
\tcblower
Bicolor
\end{tcolorbox}
\end{tcblisting}

當然除了上述的技巧之外,tcolorbox 還有許多用處我沒有講到,如果想要好好的研究可以參考他的使用手冊<連結>。