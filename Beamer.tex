\chapter{用 Beamer 做簡報}

如果想要用 \LaTeX\ 製作簡報,可以利用 beamer 這個文件類型來製作,不只可以做差一張張的靜態投影片,也可以創造出一點點的動畫效果

\subsection{基礎使用}

一個簡單的 beamaer 範例

\begin{tcblisting}{listing only}
\documentclass{beamer} %使用 beamer 為文件類型
\usepackage{xeCJK}
\setCJKmainfont{TW-Kai}
\author{Si manglam}
\title{如何使用 Beamer}
\subtitle{Beamer 大作戰}
\institute{IT 邦幫忙}
\date{2022/09/10}
\begin{document}
\begin{frame}
\titlepage
\end{frame}
\begin{frame}
\frametitle{標題}
\end{frame}
\end{document}
\end{tcblisting}

\begin{itemize}
\item frame 環境創造了新的投影片,所有要在投影片上的內容都要在這個環境內
\item \verb`\titlepage` 自動輸出標題頁
\item \verb`\frametitle`	為當前投影片加入標題
\end{itemize}

如果作者並不只有一個人且來自不同機構,我們需要用微調來讓投影片更美觀

\begin{tcblisting}{listing only}
\title[重大發現]{一個足以改變人類未來的重大發現}
\author[Luke \& manglam]{Si manglam\inst{1} \and Luke\inst{2}}
\institute[實驗室、知名大學]{\inst{1}不具名的實驗室 \and \inst{2}某間知名大學}
\date[22xx 知名會議]{22xx/xx/xx某知名會議}
\end{tcblisting}

上面的例子中,兩個作者與兩個機構中間都用\verb`\and ` 隔開,方括號內的則是在其他地方顯示的,現在可能還看不出差別,但看到下面用 \verb`\usetheme{}` 來使用 beamer 內建主題的範例就可以看出明顯的差異了

\begin{tcblisting}{listing only}
\usetheme{Madrid}
\begin{document}
\begin{frame}
\titlepage
\end{frame}
\begin{frame}
\frametitle{標題}
\end{frame}
\end{document}
\end{tcblisting}

可以看到這樣美觀了許多,同時方括號的內容顯示在下面那排,更多的主題可以在<連結>中找到

\section{小技巧}

beamer 與一般的文件一樣可以用 \verb`\tableofcontents `來建立目錄

<圖片>

也可以利用 \verb`\tableofcontents[currentsection]`來明確的標示現在的章節

<圖片>

beamer 也有針對投影片用途新定義一些環境

\begin{tcblisting}{listing only}
\begin{block}{這是一個 block}
文字
\end{block}

\begin{alertblock}{特別注意}
文字文字文字文字文字
\end{alertblock}

\begin{examples}
文字文字文字文字文字
\end{examples}
\end{tcblisting}

雖然 beamer 的主題有限,但只要換個色系就不會有人發現了

\begin{tcblisting}{listing only}
\usetheme{Madrid}
\usecolortheme{seahorse}
\end{tcblisting}

\section{overlay}

利用 \verb`\only<頁數>{}` 與 \verb`\discover<頁數>{}` 可以控制元素出現的時機,兩者差別在一個是只在特定頁數出現,一個是隱形且只會在特定頁數出現。

\begin{tcblisting}{listing only}
\begin{center}
A\only<1>{第一頁出現}\\
A\discover<1-2>{第一、二頁}\\
A\only<2-3>{第二、三頁}\\
A\discover<3>{第三頁}\\
\end{center}
\end{tcblisting}

如果是想要讓條列的內容一條條出現可以直接在用 \verb`\begin{itemize}[<+->]`

\begin{tcblisting}{listing only}
\begin{itemize}[<+->]
\item 一
\item 二
\item 三
\end{itemize} 
\end{tcblisting}

但如果要特別指定頁數,就要在 \verb`\item `後加 <頁數>

\begin{tcblisting}{listing only}
\begin{itemize}[<+->]
\item<1-> 一
\item<2> 二
\item<3-4> 三
\item<4> 四
\item<5> 五
\end{itemize} 
\end{tcblisting}

可以看到有三出現在兩頁,這些就是 beamer 如何控制 Overlay。