\chapter{繼續前行}

這篇就不再寫技術相關的內容了,而要來介紹哪裡可以找到更多關於 \LaTeX\ 的資料,畢竟書本的內容有限,且學習不能一直閉門造車。

\section{網頁}

以下幾個網頁是我很推薦的資料來源

\begin{itemize}
\item CTAN
\item Overleaf
\item Stack Exchange
\end{itemize}

CTAN 是 Comprehensive TeX Archive Network 的縮寫,基本上只要是 \TeX\ 有關的資料都會被收藏在此(\LaTeX\ 當然也被包含在內),如果有什麼 package 或使用手冊想要找,甚至是自己寫了一個 package 想要與全世界的 \TeX\ 使用者共享,只要到 CTAN 就對了。

Overleaf 不只提供了線上編譯 \LaTeX\ 的服務,他們也為了推廣 \LaTeX\ 寫了許多的技術文章,最棒的是他們的技術文章是為了初學者而設計的,所以不用怕看不懂,但想當然的內容是用英文寫的。

大部分人應該都聽過 Stack Exchange ,如果你有什麼問題想問,不仿先來這裡看看有沒有人問過。

如果有相關的問題,可以先去這些地方尋找答案。

\section{書籍}

書籍有以下幾本

\begin{itemize}
\item The \TeX\ book
\item The Not So Short Introduction to LATEX2ε
\item 簡單高效 \LaTeX
\item 大家來學 \LaTeX
\end{itemize}

The \TeX\ Book 是由高德納教授親自編寫的書籍,可以說是血統純正,但內容主要是介紹 \TeX\ 的的功能,比較適合想要了解 \LaTeX\ 系統底層的人,不建議初學者質監研讀此本書。

簡單高效 \LaTeX\ 與大家來學 \LaTeX\ 皆為中文書籍,是在資源缺乏的中文圈中為數不多的寶藏資料,兩篇皆以精簡的篇幅介紹了 \LaTeX\ 的基礎使用,並且也廣泛的介紹 \LaTeX\ 中常用的 package。 

