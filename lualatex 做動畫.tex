\chapter{LuaLaTeX 做動畫}

繼上一篇介紹了 Lua\LaTeX 後,相信大家都瞭解了 Lua\LaTeX 的基本使用方式,今天要教大家的則是如何用 Lua\LaTeX 加上 animate 製作動畫,特別提醒:這不是正常的 Lua\LaTeX 的使用方法。

\section{基礎創作}

基本上最常使用到的環境大概是物件的移動,我們不太可能一幀幀的繪製出物件的移動軌跡,因為那樣程式碼會顯得過於冗長,所以我們可以利用 for 迴圈去縮減程式碼。

\begin{tcblisting}{listing only}
\begin{luacode}
tex.sprint("\\begin{animateinline}[autoplay,loop]{10}")
for x = -4,4,0.1 do
	tex.sprint("\\begin{tikzpicture}")
	tex.sprint("\\draw[color=white] (-5,-5) rectangle (5,5);")
	tex.sprint("\\draw[fill=black] (".. x ..",0) circle (0.5);")
	tex.sprint("\\end{tikzpicture}")
	tex.sprint("\\newframe")
end
tex.sprint("\\end{animateinline}")
\end{luacode}
\end{tcblisting}

編譯出來的動畫是一個小球漸漸的從左移到右,比起一幀幀繪製,這樣簡單多了。

\section{進階創作}

更進階的創作用法可以再加上 if 迴圈,例如以下的動畫:

\begin{tcblisting}{listing only}
\begin{luacode}
tex.sprint("\\begin{animateinline}[autoplay,loop]{10}")
for x = 0,360,5 do
	tex.sprint("\\begin{tikzpicture}")
	tex.sprint("\\draw[color=white] (-2,-2) rectangle (2,2);")
	if (math.sin(math.rad(x)) > 0) then
		tex.sprint("\\draw[fill=red] (0,".. math.sin(math.rad(x)) ..") circle (0.5);")
	else
		tex.sprint("\\draw[fill=blue] (0,".. math.sin(math.rad(x)) ..") circle (0.5);")
	tex.sprint("\\end{tikzpicture}")
	tex.sprint("\\newframe")
end
tex.sprint("\\end{animateinline}")
\end{luacode}
\end{tcblisting}

編譯出的結果是一個會隨著高度變換顏色的小球,更多的使用方法就要靠你們自己去發想了,只要是有規律地動會,都可以用這種方式繪製出來的。