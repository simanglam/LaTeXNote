\chapter{etoolbox}

倒數第二天了,今天要跟大家介紹 etoolbox 這個可以讓你編輯已有命令的 package。

\section{使用之前}

因為這個 package 是讓你編輯已有命令,所以在編輯之前我們必須先找出命令的原始定義,LaTeX 提供了 \verb`\show` 這個命令來協助我們,只要在 \verb`\show`後面接上想要查詢的指令,就可以在 .log 檔中看到命令的定義,除了利用 \verb`\show` 查詢之外,我們也可以直接在原始碼中尋找定義。

不過以上兩種方式都有不便之處,\verb`\show` 無法快速的顯示出命令的定義,直接在原始碼中尋找定義又會花上大把的時間。我們急需一個更方便的方式來尋找命令的定義,latexdef 是你的一個好選擇,只要在終端機上打出 latexdef 加上你要找尋的命令即可。

\begin{tcblisting}{listing only}
latexdef TeX
\end{tcblisting}

\section{實戰}

這裡以有章節標題頁無法被自定義為例,跟章節標題有關的命令只有\verb`\chapter` 所以先讓我們看看 \verb`\chapter` 的定義:

\begin{tcblisting}{listing only}
latexdef chapter
\chapter:
undefined
\end{tcblisting}

你會發現怎麼找不到他的定義,這是因為 latexdef 預設是載入 Plain \TeX\ 格式,而不是 \LaTeX\ 格式,並且預設的文件格式是 article 所以我們還需要指定文件格式:

\begin{tcblisting}{listing only}
latexdef --tex latex -c report chapter
\chapter:
\long macro:->\if@openright \cleardoublepage \else \clearpage \fi \thispagestyle {plain}\global \@topnum \z@ \@afterindentfalse \secdef \@chapter \@schapter
\end{tcblisting}

\begin{itemize}
\item \verb`--`tex 是指定載入的格式
\item -c 是指定載入的文件類別
\end{itemize}

這樣我們就得到\verb|\chapter|的定義了,在裡面找到\verb|\thispagestyle{plain}|這個罪魁禍首,利用 etoolbox 提供的\verb|\xpatchcmd|來修改:

\begin{tcblisting}{listing only}
%\xpatchcmd{命令}{修改前內容}{修改後內容}{ }{ }
\xpatchcmd{\chapter}{\thispagestyle{plain}}{ }{ }{ }
\end{tcblisting}

這樣就可以讓標題頁的樣式是自定義的樣式了。這也是簡單修改命令定義的方式。